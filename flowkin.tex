% created using this great simple tikz flowchart example: http://www.texample.net/tikz/examples/simple-flow-chart/
% Define block styles
\tikzstyle{decision} = [diamond, draw, text width=7em, text badly centered, inner sep=0pt]
\tikzstyle{block} = [rectangle, draw, text width=10em, text centered, minimum height=4em]
\tikzstyle{rounded} = [block, rounded corners]
\tikzstyle{line} = [draw, -latex']
\tikzstyle{title} = [text width=15em, text centered]
\tikzstyle{none} = [inner sep=0]

\hspace*{-0.22\textwidth}
\begin{tikzpicture}[transform canvas={scale=0.82},node distance = 0.5 and 1, auto]
    % Place nodes
    \node [block] (init) {\textbf{Zuganfang}};
    \node [rounded, below=of init] (do1) {(Karten spielen)};
    \node [rounded, below=of do1] (door) {\textbf{T\"ur eintreten}\\T\"urkarte aufdecken};

    \node [block, below=of door] (trap) {\textbf{Falle/Fluch}\\trifft dich};
    \node [block, left=of trap] (monster) {\textbf{Monster}\\bek\"ampfen};
    \node [block, right=of trap] (crass) {\textbf{Klasse/Rasse}\\du nimmst sie an};
    \node [block, right=of crass] (other) {\textbf{Alles andere}\\auf die Hand nehmen};

    \node [rounded, below=3 of monster] (help) {Um Hilfe bitten};
    \node [rounded, below=of help] (fight) {\textbf{Kampf}};
    \node [decision, below=of fight] (eval) {Level + Items \\\textbf{$>$}(= Krieger)\\ Monsterstufe};
    \node [block, below=of eval] (loot) {\textbf{Sieg}\\Pl\"undern \& Leveln};

    \node [decision, below=of trap] (beef) {\textbf{Bist du auf \"Arger aus?}};
    \node [none, above right=0.7 of beef] (h2) {};

    \node [rounded, below=of beef, right=of fight] (newmon) {Monster von der Hand spielen};
    \node [rounded, right=of eval, below=1.7 of newmon] (do2) {Karten spielen (alle)};
    \node [decision, below= of do2] (run) {\textbf{Flucht}\\w\"urfeln + Modifikator};

    \node [block, below right=1.5 and -1.375 of run] (bad) {\textbf{Schlimme Dinge}\\treffen dich};
    \node [rounded, right=of newmon] (search) {\textbf{Raum durchsuchen}\\Ziehe verdeckte T\"urkarte auf die Hand};
    
    \node [rounded, below=3 of run] (do3) {(Karten spielen)};
    \node [circle, draw, inner sep=0, above=0.5 of do3.north] (h1) {};
    \node [rounded, below=of do3] (do4) {\textbf{Milde Gabe}\\Handkarten bis auf 5 (7 Zwerge) an Spieler mit geringster Stufe geben};

    \node [rounded, below=of do4] (end) {\textbf{Zugende}};

    \node [rounded, right=5 of run] (cheat) {\textbf{Schummeln}};
    \node [rounded, below=of cheat] (nocheat) {\textbf{Checken, dass niemand schummelt}};

    \node [title,above=4.5 of monster] (title) {\huge{\textbf{MUNCHKIN Flowchart}}};
    \node [title,below=of title] (link) {\small{elfalko/munchkinflowchart\\V 1.1}};
    % Draw edges
    \path [line] (init) -- (do1);
    \path [line] (do1) -- (door);
    \path [line] (door) -| (monster);
    \path [line] (door) -- (trap);
    \path [line] (door) -| (other);
    \path [line] (door) -| (crass);
    \path [line] (monster) -- (help);
    \path [line] (help) -- (fight);
    \path [line] (fight) -- (eval);

    \path [line] (eval.south east) -| node [near start, below] {Nein, aber ...} (do2);
    \path [line] (do2.north) |- (eval.north east);

    \path [line] (trap) -- (beef);
    \path [line] (crass.south) |- (h2) -- (beef.north east);
    \path [line] (other.south) |- (h2) -- (beef.north east);
    \path [line] (beef) -- node [] {ja} (newmon);
    \path [line] (newmon.south) |- +(-2.4,-0.5) |- (help);
    \path [line] (beef.south east) -- +(0.7,-0.7) -| node [near start, above] {nein} (search);
    \path [line] (eval) -- node [] {ja} (loot);
    \path [line] (eval.south east) -- +(0.7,-0.7) -- node [below] {Nein} +(2.4,-0.7) -- (run.north west);

    \path [line] (run.south east) -- +(0.7,-0.7) node [right] {$<$5} -- (bad);
    \path [line] (run.south west) -- +(-0.7,-0.7) |- node [near start, left] {$\geq$5} (h1) -- (do3.north);

    \path [line] (bad.south) |- (h1) -- (do3.north);
    \path [line] (loot) |- (do3);
    \path [line] (search) |- (do3);
    \path [line] (do3) -- (do4);
    \path [line] (do4) -- (end);

    \path [line] (cheat.west) -- +(-0.5,0) |- (nocheat.west);
    \path [line] (nocheat.east) -- +(0.5,0) |- (cheat.east);

\end{tikzpicture}
